%\textit{A critical analysis. For instance, the following issues can be discussed : 
%i) What are the weaknesses/strengths of the article? 
%ii) What is its originality? 
%iii) Are the technical aspects correct? 
%iv) Are the assumptions clearly stated? v) Are the numerical illustrations convincing?}
\quad Overall I believe the paper is really informative and clear regarding shocks identification using heteroskedasticity changes, 
but it does require previous knowledge about structural VARs in general to fully apprehend the subtility of identification without just-identifying assumptions (recursive identification, sign restrictions etc\dots).
\bigbreak
For my own understanding there are a few points that I would have prefered to be more clear. 
Stating the stationarity assumption and addressing cointegration issues from the start (in the general setup definition for instance) would, in my opinion, allow the authors to remove any possible ambiguity rather than only mentioning it at the end of the estimation part.
Properties of the markov process are not really addressed, whereas papers such as Foroni et al \cite{Foroni} do state that the Markov chain is aperiodic and irreducible, standard assumptions that we saw during the course for Markov-Switching types of models.
I understand that the computational and estimation part was not the main addressing point of the paper but as I still lack familiarity with the EM algorithm I felt Herwartz et Lütkepohl \cite{HL} gave more insights about what is really happening.
\bigbreak
The empirical part is really informative and well detailed to assess the uniqueness result for the covariance matrix decomposition. 
The testing procedures and explanations for their interpretation make the model much more illustrated. 
And with these examples we can fully grasp to what extent the possibility to test the just-identifying assumptions used in the literature make such model innovative and valuable.
I believe it was to remain as close as possible to the two original papers used for illustration, but the discussions from both Herwartz et Lütkepohl \cite{HL} and Sharestani et Rafei \cite{Iran} on the lag order selection for the VAR and the use of information criteria to select the number of regimes provided more insights on how to really build a MS-VAR from scratch.
In the same vein, the issue of the economic interpretation of the exactly indentified structural shocks is quickly looked over in Lanne et al but more addressed again in Herwartz et Lütkepohl \cite{HL}.
Still, it does not reduce the quality and interest of the paper as I believe it serves it purpose pretty well: the MS-VAR framework is nicely summarized, a sufficient condition ensuring exact shock identification is provided and empirical illustrations on two models from the associated literature clearly highlight the strenghs of such model.
\bigbreak
I was already interested in Structural VAR models after taking the course of Ricco Giovanni on the topic, but this project allowed me dwell more in the literature and further understand underlying constrains of just-indentifying assumptions.